\documentclass[aps,pra,letterpaper,allowtoday,onecolumn,unpublished]{quantumarticle}
\pdfoutput=1
\usepackage{xcolor}
\usepackage{amsmath}
\usepackage{amsfonts}
\usepackage{braket}
\newcommand{\Op}[1]{\ensuremath{\mathsf{\hat{#1}}}}
\newcommand{\tgt}{\text{tgt}}
\newcommand{\ii}{\mathrm{i}}
\renewcommand{\Re}{\mathrm{Re}}
\renewcommand{\Im}{\mathrm{Im}}
\newcommand{\Liouville}[0]{\mathcal{L}}
\newcommand{\identity}[0]{\mathbf{1}}
\DeclareMathSymbol{\shortminus}{\mathbin}{AMSa}{"39}
\renewcommand{\familydefault}{\sfdefault}
\definecolor{DarkBlue}{rgb}{0.1,0.1,0.5}
\definecolor{DarkRed}{rgb}{0.75,0.,0.}

\usepackage{tikz}

\usepackage[psfixbb,graphics,tightpage,active]{preview}
\PreviewEnvironment{tikzpicture}

\begin{document}
%%%%%%%%%%%%%%%%%%%%%%%%%%%%%%%%%%%%%%%%%%%%%%%%%%%%%%%%%%%%%%%%%%%%%%%%%%%%%%%%
\begin{tikzpicture}[
  x=0.62cm,
  y=0.5cm,
]
\tikzstyle{every node}+=[font= \footnotesize ]

%%% propagation boxes
% lower
\draw[color=gray!20, fill=gray!20,rounded corners=5] (2,1.1) rectangle +(14,3.55);
\node[align=center] at (9, 1.5){\raisebox{.5pt}{\textcircled{\raisebox{-.9pt} {1}}} forward-prop and storage with guess};
% upper
\draw[color=gray!20, fill=gray!20,rounded corners=5] (2,6.35) rectangle +(14,3.55);
  \node[align=center] at (9, 9.5){\raisebox{.5pt}{\textcircled{\raisebox{-.9pt} {2}}} backward-prop of extended state/gradient};

%%% forward propagation

\node (psifw1) at (3,4.2) {%
  \begin{tikzpicture}
    \draw (-0.3,0)--(0.5,-0.25)--(1.3,0);
    \node[color=DarkRed] at (0.5,0.35) {$\Psi_k(0)$};
  \end{tikzpicture}
};

\node (psifw2) at (6,4.2) {%
  \begin{tikzpicture}
    \draw (-0.3,0)--(0.5,-0.25)--(1.3,0);
    \node[color=DarkRed] at (0.5,0.35) {$\Psi_k(t_1)$};
  \end{tikzpicture}
};
\draw[->] (psifw1) edge[bend right=50] node[below]{$\epsilon_{1}$} (psifw2);

\node (psifw3) at (9,4.2) {%
  \begin{tikzpicture}
    \draw[color=gray!20] (-0.3,0)--(0.5,-0.25)--(1.3,0);
    \node[color=gray!20] at (0.5,0.25) {$\Psi_k(t)$};
    \node at (0.5,0.35) {\dots};
  \end{tikzpicture}
};
\draw[->] (psifw2) edge[bend right=50] node[below]{$\epsilon_{2}$} (psifw3);

\node (psifw4) at (12,4.2) {%
  \begin{tikzpicture}
    \draw (-0.3,0)--(0.5,-0.25)--(1.3,0);
    \node[color=DarkRed] at (0.5,0.35) {$\Psi_k(t_{\shortminus\!1})$};
  \end{tikzpicture}
};
\draw[->] (psifw3) edge[bend right=50] node[below]{$\epsilon_{\shortminus\!2}$} (psifw4);

\node (psifw5) at (15,4.2) {%
  \begin{tikzpicture}
    \node[color=DarkRed] at (0.5,0.35) {$\Psi_k(T)$};
  \end{tikzpicture}
};
\draw[->] (psifw4)  edge[bend right=50] node[below]{$\epsilon_{\shortminus\!1}$} (psifw5);

%%% backward propagation

\node (chi1) at (3,6.8) {%
  \begin{tikzpicture}
    \draw (-0.3,0)--(0.5,0.25)--(1.3,0);
    \node at (0.5,-0.35) {$\chi^{\prime}_k(0)$};
  \end{tikzpicture}
};

\node (chi2) at (6,6.8) {%
  \begin{tikzpicture}
    \draw (-0.3,0)--(0.5,0.25)--(1.3,0);
    \node at (0.5,-0.35) {$\chi^{\prime}_k(t_1)$};
  \end{tikzpicture}
};
\draw[<-] (chi1) edge[bend left=50] node[above]{$\epsilon_{1}$} node(gradn1)[below,color=DarkBlue]{$\nabla\!J_{1}$} (chi2);

\node (chi3) at (9,6.8) {%
  \begin{tikzpicture}
    \draw[color=gray!20] (-0.3,0)--(0.5,0.25)--(1.3,0);
    \node[color=gray!20] at (0.5,-0.25) {$\chi^{\prime}_k(t)$};
    \node at (0.5,-0.35) {\dots};
  \end{tikzpicture}
};
\draw[<-] (chi2)  edge[bend left=50] node[above]{$\epsilon_{2}$} node(gradn2)[below,color=DarkBlue]{$\nabla\!J_{2}$} (chi3);

\node (chi4) at (12,6.8) {%
  \begin{tikzpicture}
    \draw (-0.3,0)--(0.5,0.25)--(1.3,0);
    \node at (0.5,-0.35) {$\chi^{\prime}_k(t_{\shortminus\!1})$};
  \end{tikzpicture}
};
\draw[<-] (chi3) edge[bend left=50] node[above]{$\epsilon_{\shortminus\!2}$} node(gradn3)[below,color=DarkBlue]{$\nabla\!J_{\shortminus\!2}$}(chi4);

\node (chi5) at (15,6.8) {%
  \begin{tikzpicture}
    \draw[color=gray!20] (-0.3,0)--(0.5,0.25)--(1.3,0);
    \node[] at (0.5,-0.35) {$\tilde\chi_k(T)$};
  \end{tikzpicture}
};
\draw[<-] (chi4) edge[bend left=50] node[above]{$\epsilon_{\shortminus\!1}$} node(gradn4)[below,color=DarkBlue]{$\nabla\!J_{\shortminus\!1}$} (chi5);

%%% mu
\node (mu1) at (3,5.5) {%
  \begin{tikzpicture}
    \draw (-0.8,0.0)--(0.8,0.0);
  \end{tikzpicture}
};
\draw[->,color=DarkBlue] (mu1) -| node[right=-2pt,pos=0.65] {+} (gradn1);

\node (mu2) at (6,5.5) {%
  \begin{tikzpicture}
    \draw (-0.8,0.0)--(0.8,0.0);
  \end{tikzpicture}
};
\draw[->,color=DarkBlue] (mu2) -| node[right=-2pt,pos=0.65] {+} (gradn2);

\node (mu3) at (9,5.5) {%
  \begin{tikzpicture}
    \draw[color=white] (-0.8,0.6)--(0.8,0.6);
    \node at (0,0) {\dots};
    \draw[color=white] (-0.8,-0.6)--(0.8,-0.6);
  \end{tikzpicture}
};
\draw[->,color=DarkBlue] (mu3) -| node[right=-2pt,pos=0.65] {+} (gradn3);

\node (mu4) at (12,5.5) {%
  \begin{tikzpicture}
    \draw (-0.8,0.0)--(0.8,0.0);
  \end{tikzpicture}
};
\draw[->,color=DarkBlue] (mu4) -| node[right=-2pt,pos=0.65] {+} (gradn4);


\end{tikzpicture}

%%%%%%%%%%%%%%%%%%%%%%%%%%%%%%%%%%%%%%%%%%%%%%%%%%%%%%%%%%%%%%%%%%%%%%%%%%%%%%%%

\end{document}
